\documentclass[slovak,master]{diploma}

\usepackage[autostyle=true,czech=quotes]{csquotes} % korektni sazba uvozovek, podpora pro balik biblatex
\usepackage[backend=biber, style=iso-numeric, alldates=iso]{biblatex} % bibliografie

\ThesisAuthor{Matúš Ozaniak}

\ThesisSupervisor{Mgr. Ing. Michal Krumnikl, Ph.D.}

\CzechThesisTitle{Protokol pre komunikáciu medzi uzlami siete LoRa}

\EnglishThesisTitle{LoRa-Based Protocol for Peer-to-Peer Long-Range Communication}

\SubmissionYear{2022}

\ThesisAssignmentFileName{ThesisSpecification_OZA0016.pdf}

\Acknowledgement{TODO podakovanie}

\CzechAbstract{TODO Tohle je český abstrakt, zbytek odstavce je tvořen výplňovým textem. Naší si rozmachu potřebami s posílat v poskytnout ty má plot. Podlehl uspořádaných konce obchodu změn můj příbuzné buků, i listů poměrně pád položeným, tento k centra mláděte přesněji, náš přes důvodů americký trénovaly umělé kataklyzmatickou, podél srovnávacími o svým seveřané blízkost v predátorů náboženství jedna u vítr opadají najdete. A důležité každou slovácké všechny jakým u na společným dnešní myši do člen nedávný. Zjistí hází vymíráním výborná.}

\CzechKeywords{LoRa; Mesh; Raspberry Pi; komunikačny protokol; diplomová práce}

\EnglishAbstract{TODO This is English abstract. Lorem ipsum dolor sit amet, consectetuer adipiscing elit. Fusce tellus odio, dapibus id fermentum quis, suscipit id erat. Aenean placerat. Vivamus ac leo pretium faucibus. Duis risus. Fusce consectetuer risus a nunc. Duis ante orci, molestie vitae vehicula venenatis, tincidunt ac pede. Aliquam erat volutpat. Donec vitae arcu. Nullam lectus justo, vulputate eget mollis sed, tempor sed magna. Curabitur ligula sapien, pulvinar a vestibulum quis, facilisis vel sapien. Vestibulum fermentum tortor id mi. Etiam bibendum elit eget erat. Pellentesque pretium lectus id turpis. Nulla quis diam.}

\EnglishKeywords{LoRa; Mesh; Raspberry Pi; communication protocol; master thesis}

\AddAcronym{IoT}{Internet of Things - Internet vecí}
\AddAcronym{SF}{Spreading factor}
\AddAcronym{BW}{Bandwidth}
\AddAcronym{DR}{Data rate - rýchlosť prenosu}
\AddAcronym{CR}{Coding rate}


\addbibresource{biblatex.bib}

\begin{document}

\MakeTitlePages

\listoffigures
\clearpage

\listoftables
\clearpage

\chapter{Úvod}
V dnešnej dobe sa čoraz viac stretávame s pojmom IoT alebo internet vecí. Jedna sa o lokálne siete, zložené z fyzických zariadení, ktoré tvoria uzly siete.
Zariadenia môžu byť jednoduché senzory na monitorovanie nejakej fyzikalnej veličiny, domáce spotrebiče, vozidla, prípadne 
zariadenia, ktoré je možné ovládať na diaľku. Zariadenia tvoria sieť, v ktorej si môžu medzi sebou posielať 
dáta a informácie.

K realizacií tejto siete je potrebné mať niečo, čo by zariadenia spájalo a umožňovalo im komunikáciu. Velmi používanou technológiou
v tejto oblasti je práve technológia LoRa, ktorá umožnuje bezdrôtovú komunikáciu na veľmi veľké vzdialenosti. Spôsobov ako môžu zariadenia 
v sieti tvorenej pomocou technológie LoRa komunikovať je viacero a je to populárna téma, kde neustálne vznikajú nové návrhy rôznych riešení[1][2 TODO].
%TODO sem pridat ref na zopar protokolov... meshtastic atd.

Často sa využíva riešenie LoRaWAN, ktoré sa skladá z centrálnych uzlov pripojených k internetu a zariadení, ktoré sú pripojené k centrálnym uzlom. 
Zariadenia potom komunikujú len s centrálnym uzlom a predávajú mu svojé dáta. Centrálny uzol potom dáta posiela cez internet na nejakú službu kde 
k ním môžu uživatelia pristupovať z internetu.

Pri LoRaWAN je potrebné mať nejaký centrálny uzol a ak chceme nejaké zariadenie pripojiť do siete, musí mať dosah na daný centrálny uzol. 
Takto sme limitovaní existenciou a dosahom centrálnych uzlov, čož nieje v niektorých prípadoch užitia vhodné. Neustále vznikajú nové 
protokoly, ktoré by tieto problémy riešili [1][2][3 TODO]. %TODO ref na lorawan multihop daky.

V tejto práci sa budeme venovať návrhu a vytvoreniu protokolu, ktorý by umožnil komunikáciu medzi zariadeniami v sieti tvorenej pomocou technológie LoRa,
bez nutnej existencie centrálnych uzlov. Nami vytvorený protokol bude tvoriť sieťovú topológiu typu mesh, ktorá ma oproti hviezdicovej topológií, 
využívanej pri LoRaWAN, niekoľko výhod. Su nimi napríklad škálovatelnosť siete, kedy sa môžu zo siete odoberať alebo do nej pridávať nové zariadenia, 
bez nutnosti akejkoľvek konfigurácie na ostatných zariadeniach. Z toho vyplýva aj mobilita zariadení. Zariadenia sa môžu fyzicky pohybovať a 
pokiaľ sa nachádzajú v dosahu hocijakého iného uzla, majú prístup do siete.

\chapter{LoRa a spôsob prenosu dát}
LoRa je proprietárna technológia na bezdrôtový prenos dát za pomoci rádiovych vĺn.
Používa bezlicenčné rádiové pásma, ktoré sú odlišné medzi Európou, Amerikou a Áziou a poskytuje rádiovy prenos na veľkú vzdialenosť s nízkou spotrebou energie.
V otvorenom priestranstve môže mať rádiovy prenos dosah až 10--15 km. LoRa má však veľkú limitáciu v podobe nízkej rýchlosti prenosu dát.
Rýchlosti prenosu sa pohybujú medzi 0,3 až 37,5 kbps.

Vďaka týmto aspektom je vhodná pre použitie v IoT senzorových sieťach, kde sa často vyskytujú senzory poháňané batériami a je potrebné aby vydržali dlhú dobu 
bez výmeny batérií. Okrem toho senzory väčšinou odosielajú veľmi malý obsah dát a dáta posielaju v určitých intervaloch (napr. raz za hodinu atď.), 
takže nízka prenosová rýchlosť v tomto prípade nieje problémom.

Na prenos je použitá proprietárna chirp spread-spectrum modulácia, pri ktorej sú 
vysielané symboly a každý vysielaný symbol je reprezentovaný takzvaným chirp signálom. Tento chirp signál sa postupne posúva po intervale 
vo frekvenčnom pásme, ktoré je dané vybraným bandwidth-om. 
Ako rýchlo sa chirp posúva po frekvenčnom pásme je určené parametrom spreading factor (SF). Spreading factor taktiež vyjadruje, kolko bitov je v každom 
symbole prenesených. Pri nižšiom spreading factore sa chirp posúva po frekvenčnom pásme rýchlejšie (viď Obr. \ref{fig:spreadingfactors}) a tým sa zvyšuje dátovy prenos, 
avšak zhoršuje sa citlivosť a s tým aj použitelný dosah.

\begin{figure}
	\centering
	\includegraphics[width=0.5\textwidth]{Figures/spreading factors.png}
	\caption{Porovnanie spreading factorov. Prevzaté z \cite{spreadfactorimage}}
	\label{fig:spreadingfactors}
\end{figure}
\newpage

Existujú dva druhy chirp signálov, sú nimi up-chirp a down-chirp. Pri up-chirp signále sa prechádza z najnižšej frekvencie na najvyššiu a pri 
down-chirp naopak.

Vysielaný chirp signal je ďalej rozdelený na X častí -- tzv. chips. Koľko týchto chips jeden chirp obsahuje je závisle od vybraného spreading factoru. 
Jeden chirp ma v sebe $2^{SF}$ častí alebo chips. Vysielaný symbol sa potom skladá z cyklicky posunutého chirp signálu viď 
Obr. \ref{fig:loraModulation}, kde posun definuje hodnotu daného symbolu.
\begin{figure}
	\centering
	\includegraphics[width=0.9\textwidth]{Figures/loraModulation.png}
	\caption{Nemodulovaný a modulovaný LoRa signál. Prevzaté z \cite{loraMod}, \cite{loraMod2}}
	\label{fig:loraModulation}
\end{figure}

\section{Legislativa}
V Európe sa k frekvenčnému pásmu používaného pre LoRa viažu určité obmedzenia. 
Obmedzenia sa týkaju používania fyzického média. Regulácia určuje maximálnu povolenú dobu, v ktorú môže vysielač na daných frekvenciach vysielať 
v rámci jednej hodiny a maximálny vysielací výkon, akým môžme signal vysielať.
Určuje sa klučovací pomer, ktorý hovorí koľko percent času z nejakého celkového času môže vysielač vysielať.

Ak by zariadenie vysielalo signál po dobu jednej jednotky času z celkových 10 jednotiek času, tak kľučovací pomer by bol 10 \%.

Český Telekomunikačný úrad \cite{ctu} stanovuje vo všeobecné oprávnění č. VO-R/10/07.2021-8 \cite{vor}, že 
frekvenčné pásmo, do ktorého spadá LoRa, patrí do skupiny g. Pre túto skupinu je maximálny povolený kľučovací pomer 1 \% a maximálny 
vysielací výkon 25 mW (14 dBm). Znamená to teda, že zariadenia môžu každú hodinu vysielať maximálne po dobu 36 sekúnd.
Pre pásmo 869,40--869,65 MHz je ale udelená výnimka, ktorá umožňuje vysielať s kľučovacím pomerom 10 \% a maximálnym vysielacím výkonom 500 mW (27 dBm).

\section{LoRa parametre}
Pri používaní LoRa je nutné správne zvoliť parametre prenosu. Sú nimi frekvencia, bandwidth, spreading factor a coding rate.
Použitá frekvencia je závisla od regiónu, v ktorom sa používa, viď Tabuľka \ref{tab:ismBands}. 
V Európe je možné mimo 866 MHz pásma používať aj 433 MHz pásmo. Okrem toho existuje ešte globálne používaná frekvencia 2,4 GHz.

\begin{table}[h!]
	\centering
  \small
  \setlength\tabcolsep{6pt}
	\caption[Frekvenčne pásma používane pre LoRa]{Frekvenčne pásma používane pre LoRa}
  \begin{tabular}{c|c}
    \toprule
    Region & Frekvencia (MHz)\\
    \midrule
    Ázia & 433 \\
    Európa, Rusko, India, Afrika & 863--870 \\
    Severná Amerika & 902--928 \\
    Austrália & 915--928 \\
    Kanada & 779--787 \\
    Čína & 779--787, 470--510 \\
    \midrule
  \end{tabular}
  \label{tab:ismBands}
\end{table}


Ostatné parametre sú vyberané na základe toho ako ďaleko a ako rýchlo je potrebné dáta prenášať. Je nutné zvoliť vhodný kompromis medzi rýchlosťou prenosu 
a dosahom prenosu.

Parameter bandwidth určuje šírku pásma, v ktorom sa bude chirp posúvať. Pri vyššom bandwidthe sa zvyšuje rýchlosť prenosu, avšak znižuje sa dosah.

Spreading factor určuje koľko bitov dát môže byť prenesených v každom vysielanom symbole. To určuje ako rýchlo sa chirp posúva po frekvenčnom pásme a tym pádom 
zvyšuje alebo znižuje rýchlosť prenosu na úkor zniženia alebo zvýšenia dosahu prenosu.

LoRa obsahuje korekciu chýb spôsobených rušením. Využíva k tomu samoopravný kód -- forward error correction, pri ktorom 
sa ku dátam pridavajú korekčné bity. Tieto bity môžu byť použité na opravu chyby ak k nejakej došlo.

Parameter coding rate vyjadruje aky pomer dát, ku error korekčným dátam sa má použiť. Vyšší coding rate zabezpečí spolahlivejší prenos ak 
sa nachádzame v rušivom prostredí, ale zníži rýchlosť prenosu dát, pretože ku prenášaným dátam pridáva navyše dáta potrebné na korekciu chýb.

Rýchlosť prenosu dát (Data rate -- DR) v kbps môžme vyjadriť vzťahom:
\begin{equation}
  DR = SF * \frac{BW}{2^{SF}} * \frac{4}{4+CR}
\end{equation}

\section{LoRaWAN}
LoRa je definovaná len na fyzickej vrstve. Na používanie LoRa v IoT sieťach sú však potrebné aj vyššie vrstvy sieťového modelu.
K tomu vznikol protokol LoRaWAN, ktorý je spravovaný organizáciou LoRa Alliance \cite{lora}.

LoRaWAN definuje komunikačný protokol a architektúru sieti. Siete používaju hviezdicovú alebo hviezdu hviezd topológiu, kde 
centrálnym uzlom je LoRaWAN gateway, ktorý je pripojený k internetu. Ostatne uzly siete posielajú dáta na gateway, ktorá ich preposiela na internet.

LoRaWAN definuje tri triedy zariadení, ktoré sa v sieti vyskytujú. Sú nimi triedy A, B a C, kde do triedy A spadajú zariadenia 
väčšinou poháňané batériami, ktoré po odvysielaní svojích dát otvarajú dve prijímacie okna, v ktorých čakajú na príjem dát z gateway.
Trieda B je rozšírením triedy A. Zariadenia v tejto triede môžu otvárať dodatočné prijímacie okna v naplánovaných časových intervaloch.
Zariadenia z triedy C môžu prijímať neustále. Z tohto dôvodu majú vyššiu spotrebu energie a zvyčajne su pripojené k stálemu napájaniu.
%TODO trosku viacej nech nieje prazdna skoro cela strana pred dalsiou kapitolou

\chapter{Dostupné LoRa moduly }
Pri práci s LoRa technológiou máme na výber z rôznych modulov od rôznych výrobcov.
Moduly môžme rozdeliť podla použitia na moduly pre koncové zariadenia a moduly pre gateway.
Moduly pre koncové zariadenia obvykle dokážu prijímať a odosielať iba na jednom kanále ( kombinácia LoRa parametrov --  BW, SF a frekvencia ) súčasne a majú 
nízku spotrebu energie. Moduly pre gateway dokážu prijímať a odosielať na viacerých kanáloch súčasne a majú vyššiu spotrebu energie.

V tejto časti sa budeme zaoberať dostupnými modulmi pre koncové zariadenia.
Kedže je technológia LoRa patentovaná výrobcom Semtech \cite{semtech}, tak všetky dostupné LoRa chipy su od tohto výrobcu a moduly od iných výrobcov 
sú založené na používani týchto chipov.

\section{SX127x/SX126x}
Výrobca Semtech \cite{semtech}, prináša LoRa chipy série SX127x a SX126x. Ponukajú vysoký výkon za pomerne nízku cenu a ako sme už spomínali, ostatné LoRa moduly 
implementuju tieto chipy -- modemy a rozširujú ich o ďalšiu funkcionalitu.

\subsection{SX127x}

SX127x LoRa modemy používajú frequency hopping spread-spectrum moduláciu. Čo znamená, že viaceré vysielané signály môžu zaberať rovnaký kanál, ktorý ma vysokú ochranu proti rušeniu a zároveň majú nízku spotrebu energie. %TODO divna veta ??

Modemy používajú LoRa modulačnú techniku, patentovanú firmou Semtech. Maximálny vysielací výkon modemov je 100 mW.
Vďaka tejto modulačnej technike je možné dosiahnúť vysokej citlivosti modemov.
Výrobca uvádza citlivosť cez -137 dBm pri modemoch SX1272/73 a -148 dBm pri modemoch SX1276/78/79.

Modemy SX1272 a SX1273 ponukajú menší link budget 157 dB oproti 168 dB pri modemoch SX1276/77/78/79 a majú menší rozsah frekvenčných pásiem medzi 860 a 1020 MHz.
Okrem toho majú aj vyššiu spotrebu energie.

Pri modemoch SX1276/77/78/79 je možné vybrať frekvenčné pásma z rozsahu 137 až 1020 MHz.

\subsection{SX126x}

Modemy zo série SX126x - SX1261/62/68 sú následovníkmi modemov SX127x. Majú väčší vysielací výkon vďaka integrovanému zosilovaču a menšiu spotrebu energie. Obsahujú precízny TCXO oscilátor, 
ktorý zabezpečuje presnejšie a stabilnejšie riadenie počas prevádzky modemu. Okrem LoRa modulácie obsahujú aj G(FSK) moduláciu, ktorá je vhodná pre staršie 
prípady užitia.

Modemy taktiež obsahujú +22/+15 dBm zosilovač, vďaka ktorému majú vyšší link budget oproti modemom zo série SX127x -- 170 dBm,
takže sú optimálne pre aplikácie vyžadujúce väčší dosah alebo robustnosť.
%TODO tu sa trochu este rozpisat viac... co je link budget napr atd. este modem 1280/1281 - 2.4ghz

\begin{table}[!h]
	\centering
  \small
  \setlength\tabcolsep{6pt}
	\caption[Parametre Semtech modemov]{Parametre Semtech modemov}
  \begin{tabular}{c|c|c|c|c|c|c|c|c}
    \toprule
    Modem & Frekvencia & SF & BW (kHz) & Citlivosť & Spotreba \footnotemark[0] & Rozhranie & Výkon\footnotemark[1] & Cena\footnotemark[2]\\
    \midrule
    SX1272 & 860--1020 MHz & 6--12 & 125--500 & -137 dBm & 11,2 mA & SPI & 20 dbm & 9€ \\
    SX1273 & 860--1020 MHz & 6--9 & 125--500 & -130 dBm & 11,2 mA & SPI & 20 dbm & 7€ \\
    SX1276 & 137--1020 MHz & 6--12 & 7,8--500 & -148 dBm & 12 mA & SPI & 20 dbm & 10€ \\
    SX1277 & 137--1020 MHz & 6--9 & 7,8--500 & -139 dBm & 12 mA & SPI & 20 dbm & 7€ \\
    SX1278 & 137--525 MHz & 6--12 & 7,8--500 & -148 dBm & 12 mA & SPI & 20 dbm & 8€ \\
    SX1279 & 137--960 MHz & 6--12 & 7,8--500 & -148 dBm & 12 mA & SPI, UART & 20 dbm & 11€ \\
    \hline
    SX1261 & 150--690 MHz & 5--12 & 7,80--500 & -125 dBm & 8 mA & SPI & 15 dbm & 7€ \\
    SX1262 & 150--690 MHz & 5--12 & 7,80--500 & -125 dBm & 8 mA & SPI & 22 dbm & 8€ \\
    SX1268 & 410--810 MHz & 5--12 & 7,80--500 & -148 dBm & 4.6 mA & SPI & 22 dbm & 7€ \\
    \midrule
  \end{tabular}
\end{table}
\footnotetext[0]{Spotreba počas prijímania (mA)}
\footnotetext[1]{Maximálny vysielací výkon}
\footnotetext[2]{Cena platná ku Q4 2022}

\section{RFM9xW}
Moduly RFM95W/96W/97W/98W od výrobcu HopeRF \cite{hoperf}, používajúce SX LoRa modemy od výrobcu Semtech,
 poskytujú bezdrôtový prenos na vysokú vzdialenosť s veľkou odolnosťou voči rušeniu. 
 Jedná sa vlastne len o rozšírujucí modul, ktorý obsahuje LoRa modem a elektroniku potrebnú na riadenie Semtech modemu.
 %toto pozret https://www.disk91.com/2019/technology/lora/hoperf-rfm95-and-arduino-a-low-cost-lorawan-solution/

Existuje niekoľko verzií modulov RFM9xW, kde každá verzia používa iný semtech LoRa modem a ponúka iné parametre.

\section{STM32WL}
Mikrokontroller s loramodemom
%https://www.st.com/en/microcontrollers-microprocessors/stm32wl-series.html

\section{SAM R34/R35}
Mikrokontroller s loramodemom
%https://www.microchip.com/en-us/products/wireless-connectivity/lora-technology/sam-r34-r35

\section{HPD13A}
Modem
%Tento je v T-beame tak ho spomenut + zistit co je v tom druhom ttgo...
%Je shielded

\section {EBYTE E32}
%https://www.ebyte.com/en/product-view-news.html?id=108

% + Popisat aj zariadenia ktore pouzijem - TTGO obe, RPI, armachat....


%TODO tabulka vyteka von
\begin{table}
	\centering
  \small
  \setlength\tabcolsep{2pt}
	\caption[Parametre LoRa modulov]{LoRa moduly a ich parametre}
  \begin{tabular}{c|c|c|c|c|c|c|c}
    \toprule %TODO pridat air data rate, pridat ktory SX modem pouziva
    Modul & Frekvencia & Spreading factor & Bandwidth & Citlivosť & Spotreba počas vysielania & Zbernica & Cena(TODO do footeru *k tomuto kvartalu)\\
    \midrule
    SX1272 & 860-1020 MHz & 6-12 & 125-500 kHz & -137 dBm & 10mA & SPI & X€ \\ 
    SX1273 & 860-1020 MHz & 6-9 & 125-500 kHz & -130 dBm & 10mA & SPI & X€ \\
    SX1276 & 137-1020 MHz & 6-12 & 7.8-500 kHz & -148 dBm & 9.9mA & SPI & X€ \\
    SX1277 & 137-1020 MHz & 6-9 & 7.8-500 kHz & -139 dBm & 9.9mA & SPI & X€ \\
    SX1278 & 137-525 MHz & 6-12 & 7.8-500 kHz & -148 dBm & 9.9mA & SPI & X€ \\
    SX1279 & 137-960 MHz & 6-12 & 7.8-500 kHz & -148 dBm & 9.9mA & SPI & X€ \\
    SX1261 & TODO & TODO & TODO & TODO & 4.6mA & TODO & X€ \\
    RFM95W & 868/915 MHz & 6-12 & 7.8-500 kHz & -148 dBm & 10.3 mA & SPI & ~8€ \\
    RFM97W & 868/915 MHz & 6-9 & 7.8-500 kHz & -139 dBm & 10.3 mA & SPI & ~8€ \\
    RFM96W/RFM98W & 433/470 MHz & 6-12 & 7.8 - 500 kHz & -148 dBm & 10.3 mA & SPI & ~8€ \\
    \midrule
  \end{tabular}
\end{table}


\chapter{TODO Porovnanie existujucich rieseni}
TODO tu daky text

TODO porovnat ich voci sebe

\section{Lora mesher}
Pouziva distance vector routing protocol.
Vytvara si routovaciu tabulku, kde zaznamenava IDcka nodov, cez ktore susedne nody sa knim dostane a kolko hopov ho to bude stat.

Kazda noda drzi routing table, periodicky je updatovana cez specialny typ packetu, ktory sa posiela vsetkymi nodami v sieti. (routing packet)

Pouziva freeRtos na zabezpecenie schedulingu taskov. Rozlicne tasky sa staraju o prijatie a odoslanie packetov, iny task sa stara o samotne spracovanie packetov.

\section{Meshtastic}
Mesh siet tvorena lora modulmi. Princip fungovania je zalozeny na jednoduchom multi-hop floodingu.
Kazda node znovu odvysiela prijaty packet (pokial nedosiel maxhop na 0) az kym sa packet nedostane do destinacie napriec mesh sietou.

Pouzivane Lora moduly maju zabudovany bluetooth chip, vdaka ktoremu je mozne k modulu pripojit smarthphone, ktory sluzi ako rozhranie pre
 uzivatela. Cez aplikaciu v mobile potom vytvara a prijma spravy, ktore su cez bluetooth posielane do modulu a cez lora sa posielaju do siete.

Dosah siete sa da rozsirit cez pripojenie k oficialnemu meshtastic mqtt brokerovi. Umoznuje to tak prepojit mensie lokalne mesh siete do globalnej siete. TODO pozriet viac k tomuto

Myslienka meshtasticu spociva v tom, ze vytvara komunikacnu siet na miestach kde bezne nieje napr. mobilny signal.(V horach)

timestamp sa posiela iba v GPS datach. TODO pridat text ktomu ze sa posiela pozicia z gpsky.

\section{LoRaBlink}
Multi-hop protokol, ktory pouziva casovu synchronizaciu medzi nodami. Casova synchronizacia definuje sloty na pristup ku prenosovemnu kanalu.
Spravy sa sietou siria pomocou floodingu.

Siet sa sklada z jedneho datasinku (gateway) a viacerych nodov, ktore posielaju data do datasinku alebo data zneho prijmaju.
V urcitych intervaloch datasink vysle tzv. beacon. Tento sluzi na casovu synchronizaciu a znaci novu epochu. Kazda epocha obsahuje N 
slotov, v ktorych mozu nody vysielat data. Beacon sprava obsahuje hop count, ktora udava vzdialenost ku data-sinku.

Ked node prijme beacon signal, vysle svoj vlastny beacon signal v dalsom volnom slote, ktory vybera na zaklade vzdialenosti od data-sinku.

Ked node potrebuje poslat data, tak vybere dalsi volny slot a v nom vysiela data. Ak tieto data prijme node, ktora nieje sink a jej hop count ku sinku je mensi ako hop count
vysielajucej nody tak data v dalsom slote retransmitne. Toto sa opakuje az kym data nedosiahnu datasink. TODO dostudovat, spravit lepsi popis

- je to siet vyzadujuca jeden hlavny node (data-sink/gateway), tento node je potrebny na riadenie siete, pretoze vsetky ostatne nody sa synchronizuju na neho



\chapter{TODO Vlastna implementacia}
\chapter{TODO Testovanie vykonnosti + test voci existujucim protokolom?}


% Seznam literatury
\printbibliography[title={Literatura}, heading=bibintoc]

% Prilohy
%\appendix
%\input{Chapters/Appendix1.tex}
%\input{Chapters/Appendix2.tex}

% Priloha vlozena primo do hlavniho LaTeX souboru. Ne vsechny prilohy je nutne mit ve zvlastnich souborech.
%\chapter{Dlouhý zdrojový kód}
%\lstinputlisting[label=src:CppExternal,caption={Dlouhý zdrojový kód v jazyce C++ načtený s externího souboru}]{SourceCodes/ArraySortingAlgorithms.cpp}

\end{document}
