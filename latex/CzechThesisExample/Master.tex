% Nejprve uvedeme tridu dokumentu s volbami
\documentclass[czech,master]{diploma}
% Dalsi doplnujici baliky maker
\usepackage[autostyle=true,czech=quotes]{csquotes} % korektni sazba uvozovek, podpora pro balik biblatex
\usepackage[backend=biber, style=iso-numeric, alldates=iso]{biblatex} % bibliografie
\usepackage{dcolumn} % sloupce tabulky s ciselnymi hodnotami
\usepackage{subfig} % makra pro "podobrazky" a "podtabulky"

% Zadame pozadovane vstupy pro generovani titulnich stran.
\ThesisAuthor{Matúš Ozaniak}

\ThesisSupervisor{Mgr. Ing. Michal Krumnikl, Ph.D.}

\CzechThesisTitle{Protokol pre komunikáciu medzi uzlami siete LoRa}

\EnglishThesisTitle{LoRa-Based Protocol for Peer-to-Peer Long-Range Communication}

\SubmissionYear{2022}

\ThesisAssignmentFileName{ThesisSpecification_OZA0016.pdf}

% Pokud nechceme nikomu dekovat makro zapoznamkujeme.
\Acknowledgement{TODO podakovanie}

\CzechAbstract{TODO Tohle je český abstrakt, zbytek odstavce je tvořen výplňovým textem. Naší si rozmachu potřebami s posílat v poskytnout ty má plot. Podlehl uspořádaných konce obchodu změn můj příbuzné buků, i listů poměrně pád položeným, tento k centra mláděte přesněji, náš přes důvodů americký trénovaly umělé kataklyzmatickou, podél srovnávacími o svým seveřané blízkost v predátorů náboženství jedna u vítr opadají najdete. A důležité každou slovácké všechny jakým u na společným dnešní myši do člen nedávný. Zjistí hází vymíráním výborná.}

\CzechKeywords{LoRa; Mesh; diplomová práce}

\EnglishAbstract{TODO This is English abstract. Lorem ipsum dolor sit amet, consectetuer adipiscing elit. Fusce tellus odio, dapibus id fermentum quis, suscipit id erat. Aenean placerat. Vivamus ac leo pretium faucibus. Duis risus. Fusce consectetuer risus a nunc. Duis ante orci, molestie vitae vehicula venenatis, tincidunt ac pede. Aliquam erat volutpat. Donec vitae arcu. Nullam lectus justo, vulputate eget mollis sed, tempor sed magna. Curabitur ligula sapien, pulvinar a vestibulum quis, facilisis vel sapien. Vestibulum fermentum tortor id mi. Etiam bibendum elit eget erat. Pellentesque pretium lectus id turpis. Nulla quis diam.}

\EnglishKeywords{typography; \LaTeX; master thesis}

\AddAcronym{DVD}{Digital Versatile Disc}
\AddAcronym{TNT}{Trinitrotoluen}
\AddAcronym{UML}{Unified Modeling Language}
\AddAcronym{HTML}{Hyper Text Markup Language}
\AddAcronym{TUG}{\TeX{} Users Group}

\addbibresource{biblatex.bib}

% Novy druh tabulkoveho sloupce, ve kterem jsou cisla zarovnana podle desetinne carky
\newcolumntype{d}[1]{D{,}{,}{#1}}


% Zacatek dokumentu
\begin{document}

% Nechame vysazet titulni strany.
\MakeTitlePages

% Jsou v praci obrazky? Pokud ano vysazime jejich seznam a odstrankujeme.
% Pokud ne smazeme nasledujici dve makra.
%\listoffigures
%\clearpage

% Jsou v praci tabulky? Pokud ano vysazime jejich seznam a odstrankujeme.
% Pokud ne smazeme nasledujici dve makra.
\listoftables
\clearpage


\chapter{TOOD Dostupné LoRa moduly a spôsob prenosu dát}
LoRa je proprietárna rádiová komnikačná technika.
Používa bezlicenčné rádiové pásma, ktoré sú odlišné medzi Európou, Amerikou a Áziou a poskytuje rádiovy prenos na veľkú vzdialenosť s nízkou spotrebou energie.
V otvorenom priestranstve môže mať rádiovy prenos dosah až 10-15 km.
Vďaka tomu je vhodná pre použitie v IoT alebo senzorových sieťach, kde sa často vyskytujú senzory poháňané batériami a je potrebné aby vydržali dlhú dobu.

Na prenos je použitá proprietárna spread-spectrum modulácia odvodená od chirp spread-spectrum modulácie, pri ktorej sú 
vysielané symboly a každý vysielaný symbol je reprezentovaný takzvaným chirpom. Tento chirp sa postupne posúva po intervale 
vo frekvenčnom pásme, ktoré je dané vybraným bandwidth-om. 
Ako rýchlo sa chirp posúva po frekvenčnom pásme je určené parametrom spreading factor (SF). Spreading factor taktiež vyjadruje, kolko bitov je v každom 
symbole prenesených. Pri nižšiom spreading factore sa chirp posúva po frekvenčnom pásme rýchlejšie a tým sa zvyšuje datovy prenos, avšak zhoršuje sa citlivosť a dosah.

TODO obrazok porovnania spreading factorov

Pri používaní LoRa je nutné správne zvoliť parametre prenosu. Sú nimi frekvencia, bandwidth, spreading factor a coding rate.
Použitá frekvencia je závisla od regiónu, v ktorom sa používa. V Európe sa používa 868 MHz, v USA 915 MHz a v Ázii 923 MHz. Okrem toho existuje ešte globálne používaná 
frekvencia 2.4 GHz.

Ostatné parametre sú vyberané na základe toho ako ďaleko a ako rýchlo je potrebné dáta prenášať. Je nutné zvoliť vhodný kompromis medzi rýchlosťou prenosu a dosahom.
Pomocou LoRa dokážeme dáta prenášať rýchlosťou až 253 kbit/s, avšak pri takýchto rýchlostiach sa dramaticky zníži dosah.
%TODO overit ci 253 kbit je realnych
Parameter bandwidth určuje šírku pásma, v ktorom sa bude chirp posúvať. Pri vyššom bandwidthe sa zvyšuje rýchlosť prenosu, avšak zníži sa dosah.

Spreading factor určuje koľko bitov je v každom symbole. Zároveň určuje ako rýchlo sa chirp posúva po frekvenčnom pásme a tym pádom zvyšuje alebo znižuje rýchlosť 
prenosu na úkor zniženia alebo zvýšenia dosahu prenosu.
%TODO lepsi preklad error-correcting data
LoRa obsahuje korekciu errorov spôsobených rušením. Parameter coding rate vyjadruje pomer dát ku error-correcting dátam. Vyšší coding rate zabezpečí spolahlivejší prenos ak 
sa nachádzame v rušivom prostredí, ale zníži rýchlosť prenosu dát pretože ku každému prenášanému packetu pridáva viacej dát.

\subsection{LoRaWAN}
LoRa je definovaná len na fyzickej vrstve. Na používanie LoRa v IoT sieťach sú však potrebné aj vyššie vrstvy sieťového modelu.
K tomu vznikol protokol LoRaWAN, ktorý je spravovaný organizáciou LoRa Alliance \cite{lora}.

LoRaWAN definuje komunikačný protokol a architektúru sieti, ktoré LoRaWAN používajú. Siete používaju hviezdicovu alebo hviezda hviezd topológiu, kde 
centrálnym uzlom je LoRaWAN gateway, ktorá je pripojená k internetu. Ostatne uzly siete posielajú dáta na gateway, ktorá ich preposiela na internet.


\section{SX127x/SX126x}
\subsection{SX127x}
Výrobca Semtech \cite{semtech}, prináša LoRa modemy serie SX127x a SX126x. 

SX127x LoRa moduly používajú frequency hopping spread-spectrum moduláciu. Čo znamená, že viaceré vysielané signály zaberajú rovnaký kanál, ktorý ma 
vysokú ochranu proti rušeniu a zároveň majú nízku spotrebu energie.

Moduly používajú LoRa modulačnú techniku, patentovanú firmou Semtech. Maximálny vysielací výkon modulov je 100mW. %TODO pridat ref na patent i guess
Vďaka tejto modulačnej technike je možné dosiahnúť vysokej citlivosti modulov.
Výrobca uvádza citlivosť cez 137 dBm pri moduloch SX1272/73 a 148 dBm pri moduloch SX1276/78/79.

Moduly SX1272 a SX1273 ponukajú menší link budget - 157 dB oproti 168 dB pri moduloch SX1276/77/78/79 a majú menší rozsah frekvenčných pásiem medzi 860 a 1020 MHz.
Okrem toho majú aj vyššiu spotrebu energie.

Pri moduloch SX1276/77/78/79 je možné vybrať frekvenčné pásma z rozsahu 137 až 1020 MHz.

\subsection{SX126x}
TODO tabulka preteka

Moduly zo série SX126x - SX1261/62/68 sú následovníkmi modulov SX127x. Majú väčší vysielací výkon vďaka integrovanému zosilovaču a menšiu spotrebu energie. Obsahujú precízny TCXO oscilátor, 
ktorý zabezpečuje presnejšie a stabilnejšie riadenie počas prevádzky modulu. Okrem LoRa modulácie obsahujú aj G(FSK) moduláciu, ktorá je vhodná pre staršie 
prípady užitia.

Moduly taktiež obsahujú +22/+15 dBm zosilovač, vďaka ktorému majú vyšší link budget oproti modulom zo série SX127x - 170 dBm,
takže sú optimálne pre aplikácie vyžadujúce dosah alebo robustnosť.

\section{RFM9xW}
Moduly RFM95W/96W/97W/98W od výrobcu HopeRF \cite{hoperf}, používajúce SX LoRa modemy od výrobcu Semtech, poskytujú bezdrôtový prenos na 
vysokú vzdialenosť s veľkou odolnosťou voči rušeniu.
%TODO tabulka vyteka von
\begin{table}
	\centering
  \small
  \setlength\tabcolsep{2pt}
	\caption[Parametre LoRa modulov]{LoRa moduly a ich parametre}
  \begin{tabular}{c|c|c|c|c|c|c|c}
    \toprule %TODO pridat air data rate
    Modul & Frekvencia & Spreading factor & Bandwidth & Citlivosť & Spotreba počas vysielania & Zbernica & Cena(TODO do footeru *k tomuto kvartalu)\\
    \midrule
    SX1272 & 860-1020 MHz & 6-12 & 125-500 kHz & -117 - -137 dBm & 10mA & SPI & X€ \\ 
    SX1273 & 860-1020 MHz & 6-9 & 125-500 kHz & -117 - -130 dBm & 10mA & SPI & X€ \\
    SX1276 & 137-1020 MHz & 6-12 & 7.8-500 kHz & -111 - -148 dBm & 9.9mA & SPI & X€ \\
    SX1277 & 137-1020 MHz & 6-9 & 7.8-500 kHz & -111 - -139 dBm & 9.9mA & SPI & X€ \\
    SX1278 & 137-525 MHz & 6-12 & 7.8-500 kHz & -111 - -148 dBm & 9.9mA & SPI & X€ \\
    SX1279 & 137-960 MHz & 6-12 & 7.8-500 kHz & -111 - -148 dBm & 9.9mA & SPI & X€ \\
    SX1261 & TODO & TODO & TODO & TODO & 4.6mA & TODO & X€ \\
    RFM95W & 868/915 MHz & 6-12 & 7.8-500 kHz & -111 - 148 dBm & 10.3 mA & SPI & ~8€ \\
    RFM97W & 868/915 MHz & 6-9 & 7.8-500 kHz & -111 - 139 dBm & 10.3 mA & SPI & ~8€ \\
    RFM96W/RFM98W & 433/470 MHz & 6-12 & 7.8 - 500 kHz & -111 - 148 dBm & 10.3 mA & SPI & ~8€ \\
    \midrule
  \end{tabular}
\end{table}


\chapter{TODO Porovnanie existujucich rieseni}
TODO tu daky text

TODO porovnat ich voci sebe

\section{Lora mesher}
Pouziva distance vector routing protocol.
Vytvara si routovaciu tabulku, kde zaznamenava IDcka nodov, cez ktore susedne nody sa knim dostane a kolko hopov ho to bude stat.

Kazda noda drzi routing table, periodicky je updatovana cez specialny typ packetu, ktory sa posiela vsetkymi nodami v sieti. (routing packet)

Pouziva freeRtos na zabezpecenie schedulingu taskov. Rozlicne tasky sa staraju o prijatie a odoslanie packetov, iny task sa stara o samotne spracovanie packetov.

\section{Meshtastic}
Mesh siet tvorena lora modulmi. Princip fungovania je zalozeny na jednoduchom multi-hop floodingu.
Kazda node znovu odvysiela prijaty packet (pokial nedosiel maxhop na 0) az kym sa packet nedostane do destinacie napriec mesh sietou.

Pouzivane Lora moduly maju zabudovany bluetooth chip, vdaka ktoremu je mozne k modulu pripojit smarthphone, ktory sluzi ako rozhranie pre
 uzivatela. Cez aplikaciu v mobile potom vytvara a prijma spravy, ktore su cez bluetooth posielane do modulu a cez lora sa posielaju do siete.

Dosah siete sa da rozsirit cez pripojenie k oficialnemu meshtastic mqtt brokerovi. Umoznuje to tak prepojit mensie lokalne mesh siete do globalnej siete. TODO pozriet viac k tomuto

Myslienka meshtasticu spociva v tom, ze vytvara komunikacnu siet na miestach kde bezne nieje napr. mobilny signal.(V horach)

timestamp sa posiela iba v GPS datach. TODO pridat text ktomu ze sa posiela pozicia z gpsky.

\section{LoRaBlink}
Multi-hop protokol, ktory pouziva casovu synchronizaciu medzi nodami. Casova synchronizacia definuje sloty na pristup ku prenosovemnu kanalu.
Spravy sa sietou siria pomocou floodingu.

Siet sa sklada z jedneho datasinku (gateway) a viacerych nodov, ktore posielaju data do datasinku alebo data zneho prijmaju.
V urcitych intervaloch datasink vysle tzv. beacon. Tento sluzi na casovu synchronizaciu a znaci novu epochu. Kazda epocha obsahuje N 
slotov, v ktorych mozu nody vysielat data. Beacon sprava obsahuje hop count, ktora udava vzdialenost ku data-sinku.

Ked node prijme beacon signal, vysle svoj vlastny beacon signal v dalsom volnom slote, ktory vybera na zaklade vzdialenosti od data-sinku.

Ked node potrebuje poslat data, tak vybere dalsi volny slot a v nom vysiela data. Ak tieto data prijme node, ktora nieje sink a jej hop count ku sinku je mensi ako hop count
vysielajucej nody tak data v dalsom slote retransmitne. Toto sa opakuje az kym data nedosiahnu datasink. TODO dostudovat, spravit lepsi popis

- je to siet vyzadujuca jeden hlavny node (data-sink/gateway), tento node je potrebny na riadenie siete, pretoze vsetky ostatne nody sa synchronizuju na neho



\chapter{TODO Vlastna implementacia}
\chapter{TODO Testovanie vykonnosti + test voci existujucim protokolom?}


% Seznam literatury
\printbibliography[title={Literatura}, heading=bibintoc]

% Prilohy
%\appendix
%\input{Chapters/Appendix1.tex}
%\input{Chapters/Appendix2.tex}

% Priloha vlozena primo do hlavniho LaTeX souboru. Ne vsechny prilohy je nutne mit ve zvlastnich souborech.
%\chapter{Dlouhý zdrojový kód}
%\lstinputlisting[label=src:CppExternal,caption={Dlouhý zdrojový kód v jazyce C++ načtený s externího souboru}]{SourceCodes/ArraySortingAlgorithms.cpp}

\end{document}
