\documentclass[slovak,master]{diploma}

\usepackage{subfig}
\usepackage[autostyle=true,czech=quotes]{csquotes} % korektni sazba uvozovek, podpora pro balik biblatex
\usepackage[backend=biber, style=iso-numeric, alldates=iso]{biblatex} % bibliografie

\ThesisAuthor{Matúš Ozaniak}

\ThesisSupervisor{Mgr. Ing. Michal Krumnikl, Ph.D.}

\CzechThesisTitle{Protokol pre komunikáciu medzi uzlami siete LoRa}

\EnglishThesisTitle{LoRa-Based Protocol for Peer-to-Peer Long-Range Communication}

\SubmissionYear{2022}

\ThesisAssignmentFileName{ThesisSpecification_OZA0016.pdf}

\Acknowledgement{TODO podakovanie}

\CzechAbstract{TODO Tohle je český abstrakt, zbytek odstavce je tvořen výplňovým textem. Naší si rozmachu potřebami s posílat v poskytnout ty má plot. Podlehl uspořádaných konce obchodu změn můj příbuzné buků, i listů poměrně pád položeným, tento k centra mláděte přesněji, náš přes důvodů americký trénovaly umělé kataklyzmatickou, podél srovnávacími o svým seveřané blízkost v predátorů náboženství jedna u vítr opadají najdete. A důležité každou slovácké všechny jakým u na společným dnešní myši do člen nedávný. Zjistí hází vymíráním výborná.}

\CzechKeywords{LoRa; Mesh; Raspberry Pi; komunikačny protokol; diplomová práce}

\EnglishAbstract{TODO This is English abstract. Lorem ipsum dolor sit amet, consectetuer adipiscing elit. Fusce tellus odio, dapibus id fermentum quis, suscipit id erat. Aenean placerat. Vivamus ac leo pretium faucibus. Duis risus. Fusce consectetuer risus a nunc. Duis ante orci, molestie vitae vehicula venenatis, tincidunt ac pede. Aliquam erat volutpat. Donec vitae arcu. Nullam lectus justo, vulputate eget mollis sed, tempor sed magna. Curabitur ligula sapien, pulvinar a vestibulum quis, facilisis vel sapien. Vestibulum fermentum tortor id mi. Etiam bibendum elit eget erat. Pellentesque pretium lectus id turpis. Nulla quis diam.}

\EnglishKeywords{LoRa; Mesh; Raspberry Pi; communication protocol; master thesis}

\AddAcronym{IoT}{Internet of Things - Internet vecí}
\AddAcronym{SF}{Spreading factor}
\AddAcronym{BW}{Bandwidth}
\AddAcronym{DR}{Data rate - rýchlosť prenosu}
\AddAcronym{CR}{Coding rate}


\addbibresource{biblatex.bib}

\begin{document}

\MakeTitlePages

\listoffigures
\clearpage

\listoftables
\clearpage

\chapter{Úvod}
V dnešnej dobe sa čoraz viac stretávame s pojmom IoT alebo internet vecí. Jedna sa o lokálne siete, zložené z fyzických zariadení, ktoré tvoria uzly siete.
Zariadenia môžu byť jednoduché senzory na monitorovanie nejakej fyzikálnej veličiny, domáce spotrebiče, vozidla, prípadne 
zariadenia, ktoré je možné ovládať na diaľku. Zariadenia tvoria sieť, v ktorej si môžu medzi sebou posielať 
dáta a informácie.

K realizácií tejto siete je potrebné mať niečo, čo by zariadenia spájalo a umožňovalo im komunikáciu. Veľmi používanou technológiou
v tejto oblasti je práve technológia LoRa, ktorá umožňuje bezdrôtovú komunikáciu na veľmi veľké vzdialenosti.

Často sa využíva riešenie LoRaWAN, ktoré sa skladá z centrálnych uzlov pripojených k internetu a zariadení, ktoré sú pripojené k centrálnym uzlom. 
Zariadenia potom komunikujú len s centrálnym uzlom a predávajú mu svoje dáta. Centrálny uzol potom dáta posiela cez internet na nejakú službu kde 
k ním môžu užívatelia pristupovať z internetu.

Pri LoRaWAN je potrebné mať nejaký centrálny uzol a ak chceme nejaké zariadenie pripojiť do siete, musí mať dosah na daný centrálny uzol. 
Takto sme limitovaní existenciou a dosahom centrálnych uzlov, a hviezdicovou topológiou, čož nie je v niektorých prípadoch užitia vhodné. Neustále vznikajú nové 
protokoly, ktoré by tieto problémy riešili, napríklad za použitia mesh topológii (napr. Meshtastic \cite{meshtastic}, LoRaMesher \cite{loramesher}).

V tejto práci sa budeme venovať návrhu a vytvoreniu protokolu, ktorý by umožnil komunikáciu medzi zariadeniami v sieti tvorenej pomocou technológie LoRa,
bez nutnej existencie centrálnych uzlov. Nami vytvorený protokol bude tvoriť sieťovú topológiu typu mesh, ktorá ma oproti hviezdicovej topológii, 
využívanej pri LoRaWAN, niekoľko výhod. Su nimi napríklad škálovatelnosť siete, kedy sa môžu zo siete odoberať alebo do nej pridávať nové zariadenia, 
bez nutnosti akejkoľvek konfigurácie na ostatných zariadeniach. Z toho vyplýva aj mobilita zariadení. Zariadenia sa môžu fyzicky pohybovať a 
pokiaľ sa nachádzajú v dosahu hocijakého iného uzla, majú prístup do siete.

\chapter{LoRa a spôsob prenosu dát}
V dnešnej dobe sa na trhu nachádza veľa rôznych technológií, ktoré umožňujú bezdrôtovú komunikáciu. Niektoré technológie, ako napríklad Bluetooth, sú vhodné pre komunikáciu na malé 
vzdialenosti, napríklad komunikácia medzi mobilným telefónom a smart hodinkami. Zatiaľ čo iné technológie ako napríklad WiFi nám ponúkajú možnosť komunikácie na väčšie vzdialenosti. 
Nie všetky technológie su však vhodné pre použitie v IoT sieťach. V IoT sieťach sa často dbá na predpoklad nízkej spotreby energie. Okrem nízkej spotreby energie je taktiež vhodné 
aby mal bezdrôtový prenos veľký dosah. Z týchto dôvodov sa pri IoT sieťach využívajú takzvané LPWAN siete.
\section {LPWAN}
LPWAN (Low Power Wide Area Network) je kategória sieti s veľkou rozlohou a nízkou spotrebou energie. Tieto siete sa vyznačujú nízkymi obstarávacími 
nákladmi a dlhodobou prevádzkou. Siete sú tvorené jednoduchými a často pomerné lacnými zariadeniami, ktoré vďaka nízkej spotrebe energie dokážu pracovať dlhú dobu bez 
nutnosti pripojenia do elektrickej siete. Zariadenia bývajú napájane batériami a môžu nepretržite fungovať aj niekoľko rokov. V kombinácií so solárnymi panelmi 
môže byť ich prevádzka ešte predĺžená. 

Tieto siete sú teda vhodné pre aplikácie, kde je potrebná dlhodobá prevádzka bez nutných servisných zásahov.

LPWAN siete majú veľké pokrytie, v niektorých prípadoch to môžu byť až desiatky kilometrov v otvorenom priestranstve. Zväčša využívajú na prenos 
Sub-GHz frekvenčné pásma, ktoré nevyžadujú na používanie žiadnu rádiovú licenciu.

Technológii využívajúcich LPWAN siete je viacero. Medzi tie najznámejšie patria napríklad Sigfox \cite{sigfox}, LoRa \cite{lora}, NB-IoT a iné.

\section {LoRa}
LoRa je proprietárna technológia na bezdrôtový prenos dát za pomoci rádiových vĺn.
Používa bez-licenčné rádiové pásma, ktoré sú odlišné medzi Európou, Amerikou a Áziou, a poskytuje rádiový prenos na veľkú vzdialenosť s nízkou spotrebou energie.
V otvorenom priestranstve môže mať rádiový prenos dosah až 10--15 km. LoRa má však veľkú limitáciu v podobe nízkej rýchlosti prenosu dát.
Rýchlosti prenosu sa pohybujú medzi 0,3 až 37,5 kbps.

Vďaka týmto aspektom je vhodná pre použitie v IoT senzorových sieťach, kde sa často vyskytujú senzory poháňané batériami a je potrebné aby vydržali dlhú dobu 
bez výmeny batérií. Okrem toho senzory väčšinou odosielajú veľmi malý obsah dát a dáta posielajú iba v určitých intervaloch (napr. raz za hodinu), 
takže nízka prenosová rýchlosť v tomto prípade nie je až tak veľkým problémom.

Na prenos dát v LoRa, je použitá proprietárna frequency hopping chirp spread-spectrum modulácia -- modulácia rozprestreného spektra s preskokom frekvencií, pri ktorej sú prenášané dáta kódovane do symbolov 
a každý vysielaný symbol je prenášaný takzvaným chirp signálom, do ktorého sa daný symbol moduluje.

Chirp signál ma konštantnú amplitúdu ale mení svoju frekvenciu lineárne s časom. 
Frekvencia sa mení v rozmedzí od spodnej hranice frekvenčného pásma, po hornú hranicu frekvenčného pásma.
Po dosiahnutí hraničnej frekvencie sa frekvencia vráti na opačnú hranicu a proces sa opakuje.
Frekvenčné pásmo, v ktorom sa chirp signál prenáša je určené vybranou šírkou pásma -- bandwidth.
Graf frekvenčnej charakteristiky a postupné zvyšovanie frekvencie chirp signálu môžme vidieť na Obr. \ref{fig:loraSymbols}.
\begin{figure}
	\centering
	\includegraphics[width=0.6\textwidth]{Figures/spreading factors.png}
	\caption{Chirp signál a porovnanie rozprestieracích faktorov. Prevzaté z \cite{spreadfactorimage}}
	\label{fig:spreadingfactors}
\end{figure}

Existujú dva druhy chirp signálov, sú nimi up-chirp signál a down-chirp signál. Pri up-chirp signále sa prechádza zo spodnej hranice frekvenčného pásma do hornej hranice a pri 
down-chirp zase naopak.

\newpage
Ako rýchlo sa chirp posúva po frekvenčnom pásme - tzn. ako rýchlo chirp signál mení svoju frekvenciu, je určené parametrom rozprestierací faktor -- spreading factor (SF). 
Rozprestierací faktor zároveň vyjadruje, koľko bitov informácie je v každom symbole prenesených. Pri nižšom rozprestieracom faktore sa chirp signál posúva po 
frekvenčnom pásme rýchlejšie (viď Obr. \ref{fig:spreadingfactors}) a tým sa zvyšuje rýchlosť dátového prenosu, 
avšak zhoršuje sa citlivosť, ktorou dokáže prijímač prijať signál a dôsledkom toho je menší použiteľný dosah.

Každý chirp signál je rozdelený na X častí -- tzv. chips. Tieto chips predstavujú skoky vo vysielacej frekvencii signálu. 
Koľko týchto chips jeden chirp obsahuje je závisle od vybraného rozprestieracieho faktoru. 
Jeden chirp je rozdelený na $2^{SF}$ častí alebo chips.

Vysielaný symbol sa potom skladá z cyklicky posunutého chirp signálu, kde posun definuje hodnotu daného symbolu. 
To znamená, že vysielaný chirp nebude začínať na spodnej hranici frekvenčného pásma, ale na určitej frekvencii korešpondujúcej so symbolom, 
ktorý je modulovaný do daného chirp signálu. Viď Obr. \ref{fig:loraSymbols}.

\begin{figure}
	\centering
	\includegraphics[width=1\textwidth]{Figures/loraSymbols.png}
	\caption{Chirp signály, do ktorých boli modulované symboly. Prevzaté z TODO}
	\label{fig:loraSymbols}
\end{figure}

Celý proces vyslania správy cez LoRa sa teda skladá z nasledujúcich krokov:
\begin{enumerate}
  \item Konverzia správy do binárneho kódu
  \item Pridanie korekčných bitov slúžiacich na opravu chýb
  \item Pridanie preambuly a kontrolného súčtu, a poskladanie do LoRa paketu
  \item Modulácia bitov do chirp signálov
  \item Odvysielanie chirp signálov
\end{enumerate}
Ukážku môžme vidieť na Obr. \ref{fig:loraModulation}.

\begin{figure}
	\centering
	\includegraphics[width=1\textwidth]{Figures/loraModulation2.png}
	\caption{Proces spracovania odosielaných dát v LoRa (Binárne hodnoty su ukážkové, nezodpovedajú reálnej konverzií). Prevzaté z TODO}
	\label{fig:loraModulation}
\end{figure}

\section{LoRa paket}
Štruktúra LoRa paketu môže byť závislá od daného použitia. Bežné sa ale stretávame s LoRa paketom, zloženým z preambuly, dát a kontrolného súčtu.

Preambula slúži na synchronizáciu prijímacieho zariadenia. Je tvorená niekoľkými opakovaniami prázdneho chirp signálu, ktorý predstavuje symbol s hodnotou 0. 
Dĺžka preambuly, tzn. počet opakovaní prázdneho chirp signálu, je stanovená konfiguráciou zariadenia. Bežne sa stretávame s dĺžkou preambuly 8.
Na Obr. \ref{fig:loraModulation} môžme vidieť, že na preambulu boli použité iba 2 prázdne chirp signály -- teda dĺžka preambuly je 2.

Prijímacie zariadenie číta prijaté symboly v určitom časovom intervale -- okne. Tento interval sa ale nemusí zhodovať s intervalom, ktorý bol použitý pri vysielaní daných symbolov.
Je preto potreba synchronizovať prijímacie zariadenie aby hodnoty symbolov boli správne interpretované.

Zariadenie po prijatí nového paketu očakáva, že na začiatku bude preambula definovanej dĺžky. Ak z paketu prečíta počet symbolov rovnakej hodnoty, zhodný 
s definovanou dĺžkou preambuly, predpokladá že sa jedná o preambulu a synchronizuje čítacie okno tak aby bolo zarovnané so symbolmi preambuly. Tzn. tak, aby symboly preambuly boli 
prečítané ako symbol 0. 

Zjednodušené znázornenie procesu čítania symbolov a posun čítacieho okna môžme vidieť na Obr. \ref{fig:loraPreamble1}.

\begin{figure}[h!]
	\centering
	\includegraphics[width=1\textwidth]{Figures/preambulaSmall.png}
	\caption{Čítanie symbolov preambuly prijatého paketu a synchronizácia čítacieho okna na základe posunu preambuly.}
	\label{fig:loraPreamble1}
\end{figure}

\newpage
\section{LoRa parametre}
Pri používaní technológie LoRa je nutné správne zvoliť parametre na bezdrôtový prenos. Medzi nastaviteľné parametre patrí frekvencia, šírka pásma, rozprestierací factor a kódovací pomer.
Použitá frekvencia je závislá od regiónu, v ktorom chceme technológiu LoRa využívať, viď Tabuľka \ref{tab:ismBands}. 

V Európe je možné mimo 866 MHz pásma používať na LoRa aj 433 MHz pásmo. Okrem toho existuje ešte globálne používaná frekvencia 2,4 GHz. Tu sme ale limitovaní 
podporou daných frekvencií v nami použitom LoRa modeme. Bežné používané moduly, zväčša nemajú podporu pre 2,4 GHz frekvenčné pásmo.

\begin{table}[h!]
	\centering
  \small
  \setlength\tabcolsep{6pt}
	\caption[Frekvenčne pásma používane pre LoRa]{Frekvenčne pásma používane pre LoRa}
  \begin{tabular}{c|c}
    \toprule
    Región & Frekvenčné pásmo (MHz)\\
    \midrule
    Ázia & 433 \\
    Európa, Rusko, India, Afrika & 863--870 \\
    Severná Amerika & 902--928 \\
    Austrália & 915--928 \\
    Kanada & 779--787 \\
    Čína & 779--787, 470--510 \\
    \midrule
  \end{tabular}
  \label{tab:ismBands}
\end{table}
\newpage

Ostatné LoRa parametre sú vyberané na základe toho ako ďaleko, ako spoľahlivo a ako rýchlo je potrebné dáta prenášať. Je nutné zvoliť vhodný kompromis medzi rýchlosťou prenosu 
a dosahom prenosu.

Parameter bandwidth (BW) určuje šírku pásma, v ktorom sa bude chirp posúvať. Pri vyššej šírke pásma sa zvyšuje rýchlosť prenosu, avšak znižuje sa použiteľný dosah.
Najbežnejšie používané su šírky pásma 125 kHz, 250 kHz a 500 kHz.

Rozprestierací faktor -- spreading factor (SF) určuje koľko bitov dát bude prenesených v každom vysielanom symbole. To určuje ako rýchlo sa chirp posúva po frekvenčnom pásme a tym pádom 
zvyšuje alebo znižuje rýchlosť prenosu na úkor zníženia alebo zvýšenia dosahu prenosu. Spreading factor je vo väčšine prípadov možné zvoliť z intervalu 6--12. 
Avšak niektoré LoRa modemy umožňujú nastaviť aj nižšie hodnoty rozprestieracieho faktoru. 

Používanie rozprestieracích faktorov prináša výhodu v podobe ortogonality správ vysielaných s rôznymi rozprestieracími faktormi. 
To znamená, že prijímač dokáže správne prijať a dekódovať správu poslanú s rozprestieracím faktorom X aj 
keď sa vysielaná správa časovo prekrýva s inou vysielanou správou s iným rozprestieracím faktorom.

LoRa obsahuje korekciu chýb spôsobených rušením. Využíva k tomu samo-opravný kód -- forward error correction (FEC), pri ktorom 
sa ku prenášaným dátam pridávajú korekčné bity. Tieto bity sú potom na prijímacej strane použité na detekciu a prípadnú opravu chyby ak k nejakej došlo vplyvom rušenia.
K nastaveniu korekcie chýb slúži parameter kódovací pomer -- coding rate (CR).

V technológií LoRa máme na výber zo štyroch možností pre parameter coding rate: 4/5, 4/6, 4/7 a 4/8. 
Označenie vyjadruje pomer bitov, ktoré nesú informáciu, ku bitom, ktoré budú reálne použité na prenos informácie. 
Napríklad pri kódovacom pomere 4/5 sa na každé 4 bity informácie pridáva 1 korekčný bit.
Vyšší kódovací pomer zabezpečí spoľahlivejší prenos dát ak sa nachádzame v rušivom prostredí, avšak zníži rýchlosť prenosu dát, 
pretože ku prenášaným dátam pridáva navyše dáta potrebné na korekciu chýb.

Rýchlosť prenosu dát (Data rate -- DR) môžme vyjadriť vzťahom:
\begin{figure}[h!]
  \centering
  \[DR = SF * \frac{\frac{4}{4+CR}}{\frac{2^{SF}}{BW}} *1000 \]
  \begin{tabular}{c c}
    DR & rýchlosť prenosu dát v bitoch za sekundu\\
    BW & šírka pásma v kHz \\
    SF & rozprestierací faktor (hodnoty 6--12) \\
    CR & kódovací pomer (hodnoty 1--4) \\
  \end{tabular}
\end{figure}

\section{LoRaWAN}
Technológia LoRa je definovaná len na fyzickej vrstve. Na používanie LoRa v IoT sieťach sú však potrebné aj vyššie vrstvy sieťového modelu.
K tomu vznikol protokol LoRaWAN, ktorý je spravovaný organizáciou LoRa Alliance \cite{lora}.

LoRaWAN definuje komunikačný protokol a architektúru sieti. Siete používajú hviezdicovú topológiu, prípadne topológiu hviezdu hviezd, kde 
centrálnym uzlom je LoRaWAN brána -- gateway, ktorá je pripojená k internetu a pevnému napájaniu. Ostatne uzly siete posielajú dáta do tejto brány, 
ktorá ich potom preposiela na internet. Tam su už dáta dostupné odkiaľkoľvek.

LoRaWAN definuje tri základne triedy zariadení, ktoré sa v sieti vyskytujú. Triedy špecifikujú funkciu zariadenia a jeho vlastnosti.
Sú nimi triedy A, B a C, kde do triedy A spadajú zariadenia väčšinou poháňané batériami, ktoré po odvysielaní svojich dát otvárajú dve prijímacie okna, 
v ktorých čakajú na príjem dát z brány.
Trieda B je rozšírením triedy A. Zariadenia v tejto triede môžu otvárať dodatočné prijímacie okna v naplánovaných časových intervaloch.
Zariadenia z triedy C môžu prijímať neustále. Z tohto dôvodu majú vyššiu spotrebu energie a zvyčajne su pripojené k stálemu napájaniu.

\section{Legislatíva}
V Európe sa k frekvenčnému pásmu používaného v technológií LoRa viažu určité obmedzenia. 
Obmedzenia sa týkajú používania fyzického média. Regulácia určuje maximálnu povolenú dobu, v ktorú môže vysielač na daných frekvenciách vysielať 
v rámci jednej hodiny a maximálny vysielací výkon, akým môže signal vysielať.

Určuje sa takzvaný klučovací pomer, ktorý hovorí koľko percent času z nejakého celkového času môže vysielač vysielať.
Ak by zariadenie vysielalo signál po dobu jednej jednotky času z celkových 10 jednotiek času, tak kľučovací pomer by bol 10 \%.

Český Telekomunikačný úrad \cite{ctu} stanovuje vo všeobecné oprávnění č. VO-R/10/07.2021-8 \cite{vor}, že 
frekvenčné pásmo, do ktorého spadá technológia LoRa, patrí do skupiny g. Pre túto skupinu je maximálny povolený kľučovací pomer 1 \% a maximálny 
vysielací výkon 25 mW (14 dBm). Znamená to teda, že zariadenia môžu každú hodinu vysielať maximálne po dobu 36 sekúnd.
Pre pásmo 869,40--869,65 MHz je ale udelená výnimka, ktorá umožňuje vysielať s kľučovacím pomerom 10 \% a maximálnym vysielacím výkonom 500 mW (27 dBm).


\chapter{Dostupné LoRa moduly }
Pri práci s technológiou LoRa máme na výber z rôznych modulov od rôznych výrobcov.
Moduly môžme rozdeliť podla použitia na moduly pre koncové zariadenia a moduly pre gateway.
Moduly pre koncové zariadenia obvykle dokážu prijímať a odosielať iba na jednom kanále ( kombinácia LoRa parametrov --  BW, SF, CR a frekvencia ) súčasne a majú 
nízku spotrebu energie. Moduly pre gateway dokážu prijímať a odosielať na viacerých kanáloch súčasne ale majú vyššiu spotrebu energie.

V tejto časti sa budeme zaoberať dostupnými modulmi pre koncové zariadenia.
Keďže je technológia LoRa patentovaná výrobcom Semtech \cite{semtech}, tak všetky dostupné LoRa čipy su práve od tohto výrobcu a moduly od iných výrobcov 
sú založené na používaní týchto čipov.

\section{SX127x/SX126x}
Výrobca Semtech \cite{semtech}, prináša LoRa čipy série SX127x a SX126x. Ponúkajú vysoký výkon za pomerne nízku cenu a ako sme už spomínali, ostatné LoRa moduly 
iba implementujú tieto LoRa čipy -- modemy a rozširujú ich o ďalšiu funkcionalitu.

\subsection{SX127x}
SX127x LoRa modemy používajú LoRa modulačnú techniku frequency hopping spread-spectrum, patentovanú firmou Semtech.

Maximálny vysielací výkon týchto modemov je 100 mW.
Vďaka použitej modulačnej technike je možné dosiahnuť vysokej citlivosti modemov.
Výrobca uvádza citlivosť cez -137 dBm pri modemoch SX1272/73 a -148 dBm pri modemoch SX1276/78/79.

%TODO vysvetlivka linkbudget do footnote
Modemy SX1272 a SX1273 ponúkajú menší link budget 157 dB oproti 168 dB pri modemoch SX1276/77/78/79 a majú menší rozsah frekvenčných pásiem medzi 860 a 1020 MHz.
Okrem toho majú aj vyššiu spotrebu energie.

Pri modemoch SX1276/77/78/79 je možné vybrať frekvenčné pásma z rozsahu 137 až 1020 MHz.

\subsection{SX126x}

Modemy zo série SX126x - SX1261/62/68 sú nasledovníkmi predošlých modemov SX127x. Majú väčší vysielací výkon vďaka integrovanému zosilňovaču a menšiu spotrebu energie. 
Obsahujú precízny TCXO oscilátor, ktorý zabezpečuje presnejšie a stabilnejšie riadenie počas prevádzky modemu. Okrem LoRa modulácie obsahujú aj G(FSK) moduláciu, ktorá je vhodná pre staršie 
prípady použitia.

Modemy obsahujú +22/+15 dBm zosilňovač, vďaka ktorému majú vyšší link budget oproti modemom zo série SX127x. 
Ten je pri modemoch série SX126x výrobcom udávaný na 170 dBm, takže sú optimálne pre aplikácie vyžadujúce väčší dosah alebo robustnosť.
%TODO este pripadne modem 1280/1281 - 2.4ghz

\begin{table}[!h]
	\centering
  \small
  \setlength\tabcolsep{6pt}
	\caption[Parametre Semtech SX modemov]{Parametre Semtech SX modemov}
  \begin{tabular}{c|c|c|c|c|c|c|c|c}
    \toprule
    Modem & Frekvencia & SF & BW (kHz) & Citlivosť & Spotreba \footnotemark[1] & Rozhranie & Výkon\footnotemark[2] & Cena\footnotemark[3]\\
    \midrule
    SX1272 & 860--1020 MHz & 6--12 & 125--500 & -137 dBm & 11,2 mA & SPI & 20 dbm & 9€ \\
    SX1273 & 860--1020 MHz & 6--9 & 125--500 & -130 dBm & 11,2 mA & SPI & 20 dbm & 7€ \\
    SX1276 & 137--1020 MHz & 6--12 & 7,8--500 & -148 dBm & 12 mA & SPI & 20 dbm & 10€ \\
    SX1277 & 137--1020 MHz & 6--9 & 7,8--500 & -139 dBm & 12 mA & SPI & 20 dbm & 7€ \\
    SX1278 & 137--525 MHz & 6--12 & 7,8--500 & -148 dBm & 12 mA & SPI & 20 dbm & 8€ \\
    SX1279 & 137--960 MHz & 6--12 & 7,8--500 & -148 dBm & 12 mA & SPI, UART & 20 dbm & 11€ \\
    \hline
    SX1261 & 150--690 MHz & 5--12 & 7,80--500 & -125 dBm & 8 mA & SPI & 15 dbm & 7€ \\
    SX1262 & 150--690 MHz & 5--12 & 7,80--500 & -125 dBm & 8 mA & SPI & 22 dbm & 8€ \\
    SX1268 & 410--810 MHz & 5--12 & 7,80--500 & -148 dBm & 4.6 mA & SPI & 22 dbm & 7€ \\
    \midrule
  \end{tabular}
\end{table}
\footnotetext[1]{Spotreba počas prijímania (mA)}
\footnotetext[2]{Maximálny vysielací výkon}
\footnotetext[3]{Cena platná ku Q4 2022}

\section{RFM9xW}
Moduly RFM95W/96W/98W od výrobcu HopeRF \cite{hoperf} implementujú SX LoRa modemy od výrobcu Semtech.
Jedná sa o jednoduchý modul, ktorý obsahuje všetku riadiacu elektroniku potrebnú pre riadenie Semtech LoRa modemu.
Okrem riadiacej elektroniky obsahuje modul aj zosilňovač signálu (+14 dBm), ktorý zvyšuje dosah bezdrôtového prenosu.

Existuje niekoľko verzií modulov RFM9xW, kde každá verzia používa iný semtech LoRa modem a zdieľa parametre daného modemu.

\section{Moduly a zariadenia použité v tejto práci}
Na vývoj a testovanie tejto práce boli použité rôzne zariadenia s rôznymi platformami. Na simulovanie jednoduchých koncových zariadení, 
ktoré môžu predstavovať napríklad nejaký senzor, boli použité mikrokontroléry TTGO od výrobcu LILYGO \cite{lilygo}.

Okrem mikrokontrolérov TTGO boli použité aj mikrokontroléry Raspberry Pi Pico a jednodoskový počítač Raspberry Pi.
Nami použité mikrokontroléry môžu byť neskôr rozšírene o batériu a byť mobilné.

\subsection{TTGO LoRa32}
Tento mikrokontrolér je založený na module ESP32. Obsahuje Wi-Fi, bluetooth a LoRa moduly. 
Konkrétne používa SX1276 LoRa modem.

Dá sa prepnúť do úsporného režimu spánku, ktorý znižuje spotrebu mikrokontroléru na 0,6 mA.
Mikrokontrolér obsahuje aj 0,96 palcový čiernobiely displej, ktorý je pripojený cez I2C rozhranie.

\subsection{TTGO T-Beam}
Tento mikrokontrolér je taktiež založený na module ESP32 a obsahuje Wi-Fi, bluetooth a LoRa modul.
Rovnako používa SX1276 LoRa modem no okrem spomínaných modulov obsahuje aj GPS modul.

Mikrokontrolér ma na zadnej strane držiak na batériu, z ktorej môže byť napájaný a taktiež aj nabíjací modul 
pre bezpečné nabíjanie a ochranu lítiových batérií.
V režime spánku ma spotrebu 0,2 mA.

Podporuje pripojenie dodatočného displeju, avšak nami používaný mikrokontrolér tento displej nemá.

Na Obr \ref{fig:ttgo-moduly} môžeme vidieť oba mikrokontroléry TTGO.
\begin{figure}[h!]
  \centering
  \subfloat[\centering TTGO LoRa32]{{\includegraphics[width=0.4\textwidth]{Figures/LILYGO-TTGO-LoRa32-V1.png} }}
  \qquad
  \subfloat[\centering TTGO T-beam]{{\includegraphics[width=0.35\textwidth]{Figures/ttgo-tbeam.jpg} }}
  \caption{Mikrokontroléry TTGO}
  \label{fig:ttgo-moduly}
\end{figure}

\subsection{Raspberry Pi Pico RP2040}
Mikrokontrolér od známej organizácie Raspberry Pi, založený na dvoj-jadrovom Arm procesore. 
Existuje verzia s Wi-Fi modulom aj bez. Na programovanie sa využíva jazyk C alebo MicroPython, 
prípadne derivát MicroPythonu -- CircuitPyhon. V tejto práci bude použitý práve CircuitPython.

Avšak aby bolo možné využívať tieto mikrokontroléry s technológiou LoRa, bolo potrebné pridať k nim nejaký LoRa modul.

%TODO armachat footnote?
Na implementáciu v tejto práci boli preto zvolené zariadenia Armachat, ktoré kombinujú mikrokontrolér Raspberry Pi Pico s LoRa modulom RFM96W.
Okrem toho pridávajú dvoj palcový farebný displej a klávesnicu.

Ako toto zariadenie vyzerá môžme vidieť na Obr \ref{fig:armachat}.

\begin{figure}[h!]
	\centering
	\includegraphics[width=0.4\textwidth]{Figures/armachat.jpg}
	\caption{Zariadenie Armachat}
	\label{fig:armachat}
\end{figure}

\subsection{Raspberry Pi 2B}
Jednodoskový štvor-jadrový počítač vytvorený organizáciou Raspberry Pi.

Zariadenie obsahuje eternetový port, ktorý môže slúžiť na pevné pripojenie do internetovej siete.
Počítač sam o sebe neobsahuje žiaden modul a tak pre prácu s LoRa bude pridaný RFM96W modul pripojený do vstupno-výstupných pinov počítača.

Na rozdiel od predchádzajúcich mikrokontrolérov disponuje tento jednodoskový počítač oveľa väčším výkonom a pamäťou, 
avšak ma oveľa vyššiu spotrebu energie. Z tohto dôvodu bude tento počítač plniť rolu nemobilného uzlu v sieti, ktorý bude pripojený cez ethernet do internetovej siete.


\chapter{Existujúce riešenia}
Téma mesh sietí je v tejto dobe velmi aktuálna a vývojári sa snažia vytvoriť rôzne riešenia, ktoré by boli vhodné pre rôzne účely.
Existujú rozvinuté projekty ako napríklad Meshtastic, ktorý sa snaží vytvoriť rozsiahlu decentralizovanú mesh sieť za použitia lacných zariadení.

Ďalším zaujímavým projektom je Armachat \cite{armachat}, ktorý ponúka možnosť komunikácie v prípade nedostupnosti ostatných sieti, napríklad po nejakej prírodnej alebo inej katastrofe.
Súčastou projektu Armachat sú dostupné návhry plošných spojov, ktoré implementujú Raspbery Pi Pico mikrokontroléry, RFMx LoRa moduly, display a ďalšie vymoženosti.
Výsledkom poskladania týchto plošných spojov je malé mobilné zariadenie s klávesnicou a displejom, ktoré sa dá použiť na komunikáciu prostredníctvom LoRa siete.
Originálny projekt avšak plne nepodporuje využívanie mesh topologie. Používa ale rovnakú štruktúru správ ako projekt Meshtastic a vďaka tomu je možné 
v Armachate využívať mesh sieť projektu Meshtasticu.

\section{Lora mesher}
LoRaMesher \cite{loramesher} je knižnica, ktorú je možné použiť na komunikáciu cez LoRa mesh sieť.

Na smerovanie v sieti sa používa distance vector routing protocol. Tento protokol vyberá cestu, kadiaľ pôjde správa od odosielateľa k príjemcovi, na základe 
najlepšej cesty. Najlepšiu cestu definuje ako cestu s najmenším počtom hopov -- preskokov medzi uzlami v mesh sieti.

K realizacií distance vector smerovania je potrebné mať smerovaciu tabuľku, ktorá obsahuje informácie o ID uzlov, cez ktoré susedné uzly sa dajú dosiahnuť a 
koľko preskokov bude potrebných na dosiahnutie týchto uzlov. Smerovacia tabuľka je periodicky aktualizovaná cez špeciálny typ správ, ktoré sú odosielané 
všetkými uzlami v sieti. Túto smerovaciu tabuľku si drží každý uzol v sieti.

LoRaMesher používa FreeRTOS, čož je operačný systém reálneho času. Takéto operačné systémy garantujú dokončenie úloh v určitom čase.
FreeRTOS je použitý na zabezpečenie plánovania úloh. Rozličné úlohy sa starajú o prijatie a odoslanie packetov, iné úlohy sa starajú o samotne 
spracovanie prijatých packetov.

LoRaMesher dokáže nájsť novo vytvorené uzly v sieti vďaka smerovaciemu protokolu. Pri odoslaní správ čaká na prijatie ACK správy, ktorá potvrdzuje, 
že správa bola prijatá a tým zaistuje spoľahlivosť. Správy väčšie ako 222 bajtov rozdeluje na viacero správ, ktore pošle postupne.

\section{Meshtastic}
Meshtastic \cite{meshtastic} vytvára mesh sieť za použitia lacných mikrokontrolérov s LoRa modulmi.
Myšlienka tohto projektu spočíva v tom, že vytvára komunikačnú sieť na miestach kde neexistuje spoľahlivá infraštruktúra na bezdrôtovú komunikáciu (napr. v horách).

Posielanie správ v sieti je založené na jednoduchom multi-hop floodingu.
Každý uzol znovu odvysiela packet, ktorý prijal ( pokial nedošiel počet preskokov na 0 ), až kým sa packet nedostane do určenej destinácie naprieč mesh sietou.
Prenášane správy su šifrované.

Používane zariadenia s LoRa modulmi majú zabudovaný bluetooth chip, vďaka ktorému je možne sa k zariadeniu pripojit cez smartphone, ktorý slúži ako rozhranie pre 
užívateľa. Cez aplikáciu v smartphone môže používateľ vytvárať a prijímať správy. Správy sa cez bluetooth prenášaju zo smartphone do zariadenia a odošlu sa cez 
LoRa do siete. %TODO trosku preformulovat

Meshtastic poskytuje možnosť pripojenia sa k oficiálnemu meshtastic mqtt brokerovi. Toto umožňuje prepojiť malé lokálne mesh siete do väčšej globálnej siete a 
tak rozšíriť dosah siete.

\section{LoRaBlink}
Multi-hop protokol, ktorý požíva časovú synchronizáciu medzi uzlami. Časová synchronizácia definuje sloty, v ktorých môže uzol pristupovať ku prenosovému médiu a 
vysielať svojé dáta. Správy sa sieťou šíria pomocou multi-hop floodingu.

Sieť sa skladá z jedneho datasinku (gateway) a viacerých uzlov. Uzly sieťe posielajú data do datasinku alebo data z neho prijímaju.
V určitých intervaloch datasink vyšle tzv. beacon signál. Tento signál slúži na časovú synchronizáciu medzi uzlami a znači začiatok novej epochy. 
Každá epocha obsahuje N slotov, v ktorých môžu uzly vysielať data. Beacon správa obsahuje hop count, ktorá udáva vzdialenosť ku datasinku.
Ked nejaký uzol príjme beacon signál, vyšle svoj vlastný beacon signál v dalšiom volnom slote, ktorý vyberá na základe vzdialenosti od datasinku (hop count).

Ked uzol potrebuje poslať nejaké data, tak vyberie najskorší volný slot a v nom vysiela data. Ak tieto data príjme uzol, ktorý nieje datasink a 
hop count daného uzlu ku sinku je menší ako hop count vysielajucého uzlu, tak data v dalšiom slote znovu prepošle. Toto sa opakuje, až 
kým data nedosiahnu datasink.
%TODO dostudovat, rozsirit popis

Takto tvorená sieť avšak vyžaduje existenciu nejakého hlavného uzlu (datasinku), ktorý je potrebný na riadenie siete prostredníctvom časovej synchronizácie.

\section{Pymesh}
Pymesh je súčasťou Pycom \cite{pycom} ekosystému. Tento ekosystém je určený na vývoj IoT systémov. Ponúka rôzne zariadenia, ktoré su určené na použitie s týmto 
ekosystémom. Zariadenia WiFi, bluetooth a LoRa. Zariadenia je možné rozšíriť o rôzne moduly so senzormi, ktoré rozšírujú ich schopnosti. %TODO preformulovat

PyMesh sieť sa skladá z uzlov, ktoré môžu vystupovať v roli gateway alebo bežného uzla. Uzly typu gateway su označované ako Border Routers a prepájajú LoRa sieť s 
internetom. Uzly v sieti môžu komunikovať ad-hoc. V sieti môže dojsť k situacií kedy bude chcieť viacero uzlov vysielať v rovnakom čase a došlo by tak ku kolizií.
Aby sa zabránilo takýmto situáciam, je použitá metóda Listen Before Talk, pri ktorej sa pred vysielaním signálu overí, či nie je v sieti už niekto iný, kto by 
vysielal. Pokial je, tak sa signál vyšle neskôr.

PyMesh je primárne určený na použitie s Pycom zariadeniami a použitie na inom zariadení by vyžadovalo väčšie úpravy zdrojového kódu.

\section{Synchronous LoRa Mesh}
Tento projekt\cite{synchronouslorameshnetwork} vznikol z potreby získavania real-time dát z podzemných kanalizácií. Tieto dáta sú potrebné k monitorovaniu a predikcií kritických situacií akými 
sú napríklad záplavy.

LoRaWan však nema moc velký dosah do podzemia. Z toho dôvodu by bolo potrebné v podzemných priestoroch implementovať LoRaWan gateway-e, ktoré su ale energeticky náročne, drahé a vyžadujú 
pevné pripojenie do elektrickej a prípadne aj internetovej siete. Okrem toho, by museli byť všetky ostatné uzly v podzemnej sieti v dosahu gateway a pri väčšej podzemnej sieti 
by teda bolo potrebné implementovať viacero LoRaWan gateway.

Tento projekt sa snaží vyriešiť tieto problémy. Prináša protokol, ktorý rozširuje LoRaWan o tzv. repeater uzly (RN) viď Obr. \ref{fig:synchronouslora}. Tieto uzly sa vyskytujú na povrchu a preposielajú dáta z 
podzemných monitorovacích uzlov (sensor node  - SN) do LoRaWan siete. Monitorovacie uzly pod zemou tvoria mesh sieť a RN plní funkciu riadiaceho uzlu pre podzemnú mesh siet. 
Komunikácia medzi RN a SN je synchronizovaná pomocou presného časovania, čo umožňuje koordináciu zmeny stavov SN z režimu spánku do normálneho režimu v čase, kedy 
potrebuje SN prijímať a odosielať dáta. Komunikácia sa cez uzly šíri multi-hop prístupom, za použitia smerovacej tabulky.

Protokol používa na riadenie komunikácie metódu TDMA. RN priradí každej SN časový slot, v ktorom SN môže vyslať alebo prijať dáta.
Monitorovacie uzly su väčšinu času v režime spánku, zobudia sa len v ich určenom časovom slote a vďaka tomu maju tieto uzly nízku spotrebu energie.

Novo pripojený uzol do sieťe musí čakať na periodicky beacon, vysielaný RN uzlom. Všetky uzly si držia v sebe smerovaciu tabuľku. Po pripojení nového uzla sa sieťou sa propaguje 
správa na aktualizovanie smerovacej tabuľky.

\begin{figure}
	\centering
	\includegraphics[width=0.6\textwidth]{Figures/synchronouslorameshnetwork.png}
	\caption{Schéma Synchronous Lora Mesh. Prevzaté z \cite{synchronouslorameshnetwork}}
	\label{fig:synchronouslora}
\end{figure}

\section{Porovnanie voči nášmu riešeniu}
Niektoré zo spomenutých riešení využívaju smerovacie tabuľky na efektívnejši prenos dát v rámci siete. Smerovacie tabuľky môžu byť však limitáciou pokiaľ 
chceme dosiahnuť mobility uzlov. V tom prípade je potrebné zabezpečiť dostatočne časté aktualizácie smerovacích tabuliek.

Niektoré spomenuté riešenia limitujú komunikáciu medzi uzlami na vyhradené časové okná, mimo ktoré su uzly v úspornom režime. To môže byť 
veľkou limitacou pri niektorých aplikáciách, kde je potrebná komunikácia v reálnom čase  (napr. chat).

Protokol navrhnutý v tejto práci nebude využívať žiadne smerovacie tabuľky. Vďaka tomu dosiahneme mobility uzlov bez potreby 
periodických aktualizacií smerovacích tabuliek. Protokol nebude používať žiadnu časovú synchronizáciu ani časové okná vyhradené na komunikáciu, 
uzly siete tak budu môcť prijímať a odosielať dáta hocikedy.

Funckionalita navrhnutého protokolu bude podobná ako projekt Meshtastic, kedy sa správa v sieti preposiela až kým nedorazí do destinácie.
Správa sa odošle a uzol, ktorý túto správu prijal ako posledný ju prepošle dalej. Toto sa opakuje na dalších uzloch až kým sa správa nedostane do destinácie alebo kym správe nedôjde hoplimit.

Ak nieje destinácia v dosahu, tak o správu prídeme. Tento problém vyriešime tak, že odosielatel si bude odoslané správy ukladať a v prípade, že sa nepodarí správu doručiť, môže ju opätovne 
odoslať niekedy neskôr.

Navrhnutý protokol nieje závisly na žiadnych špeciálnych uzloch typu gateway, prípadne riadiacich uzloch. Každý uzol v sieti môže byť primitivným uzlom, ktorý bude prijímať a preposielať dáta.
Vďaka tomu môžu byť uzly realizované prostredníctvom lacných a malých zariadeni.

Protokol bude možné používať na rôznych platformách. V tejto práci vznikne implementácia protokolu pre mikrokontroléry používajúce programovací jazyk C++ a 
mikrokontroléry alebo mikropočítače podporujúce microPython/Python.

\chapter{TODO Vlastna implementacia}
\section{Navrh packetu}
header, vysvetlenie jednotlivych poli, crc, hoplimit,
\subsection{Typy sprav XXXX}
popis typov sprav
textova sprava, sensor data sprava, ping sprava, ACK sprava

\section{Funkcionalita protokolu}
dlhy popis funkcionality protokolu, ako funguje, priebeh odoslania spravy, priebeh sirenia spravy cez ostatne node,
vysvetlenie statov sprav (sent, rebroadcasted, acked atd)
vysvetlenie message queue, timeouty, retry atd
alternativne scenare, napr final node mimo dosahu, zla topologia siete atd

\section{Navrh a implementacia web gui}
popis co bude vo webgui, popis funkci, na rpi bude aj monitorovacia zalozka
mozno poriesit cachovanie dat v localstoragi -- to asi prinesie viac problemov ako vyhod...
popis implementacie, kedze webgui bude aj na slabych ttgo zariadeniach, bude treba dost optimalizovat kod webu

\section{implementacia na micropythone}
jak prebiehala implementacia, ukazky kodu atd.

\section{implementacia na pythone}
rozsirenie a uprava implementacie na micropythonupythone, ukazky kodu atd.
ak bude cas rozbehat na rpi grafanu na zobrazenie monitorovacich dat

\section{implementacia c++++}
jak prebiehala implementacia, ukazky kodu atd.
porovnanie voci micro/pythonu + rozdiely, problemy ktore sa vyskytli, ako sa to riesilo, mozno porovnanie rychlosti kodu ?
poriesit kde a akobude ulozeny kod webu... dlhy string ??

\section{Testovanie funkcnosti + medziplatformova komunikacia}
test ze to funguje, rozmiestnenie zariadeni, posielanie sprav, simulovanie sensorovych uzlov atd

\chapter{Testovanie vykonnosti + test voci existujucim protokolom?}
Test bude mozny asi len voci meshtasticu a to len na ttgo zariadeniach. Tie mam len dve takze to bude slaby test
meshtastic ponuka simulator kde by mohlo byt mozne zmenit kod nody a simulovat vnom moj protokol
V meshtasticu nebude fungovat mesh siet ak bude mat tvar V, moj protokol to ale ma vyriesene tak to spomenut
meranie casu dorucenia spravy pri rovnakych podmienkach, parametroch atd

Ak bude cas rozsirit armachat o moj protokol. Upravit armachat shitcode bude ale casovo narocne a vo vysledku pride armachat o moznost pouzit ho s meshtasticom.

% Seznam literatury
\printbibliography[title={Literatura}, heading=bibintoc]

% Prilohy
%\appendix
%\input{Chapters/Appendix1.tex}
%\input{Chapters/Appendix2.tex}

% Priloha vlozena primo do hlavniho LaTeX souboru. Ne vsechny prilohy je nutne mit ve zvlastnich souborech.
%\chapter{Dlouhý zdrojový kód}
%\lstinputlisting[label=src:CppExternal,caption={Dlouhý zdrojový kód v jazyce C++ načtený s externího souboru}]{SourceCodes/ArraySortingAlgorithms.cpp}

\end{document}
